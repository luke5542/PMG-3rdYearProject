% This is the final report for my 3rd year project regarding procedural
% content generation for game maps.

\documentclass[12pt,a4paper]{article}

\usepackage{pslatex}

\usepackage{url}

\title{Procedural Content Generation: Game Maps}

\author{Devin Ridgway}
%Date started: 10/10/2014

\begin{document}

\maketitle

Giving computers the ability to display more than just simple text has always been a drive in the field of Computer Science. Not only does this ability allow scientists to visualize a wide variety of data, but it creates its own market for consumer electronics. Drawing simple graphs or small pictures was the beginning, but since then we have created machines with the ability to render entire planets worth of scenery that users can immerse themselves in.

\section{The Article}

At first, the intent behind creating these detailed computer simulations was to enable a convenient way to view scenes like the design of a building or the specifications of building blocks like screws and cogs. Other motivations were targeting the recreations of our world in a similar, or even more realistic, way to famous paintings.


\newpage
\begin{thebibliography}{2}

\bibitem{cry3}
  CryEngine 3,
  Retrieved on \today\ from http://mycryengine.com/index.php?conid=53

\bibitem{evolution}
  Gamingbolt,
  \emph{Mind Blown: Video Games Graphics Evolution In The Last 30 Years},
  Retrieved on \today\ from http://en.wikipedia.org

\end{thebibliography}


\end{document}
